% thanks to http://tex.stackexchange.com/a/47579/71109
\usepackage{ifxetex}
\usepackage{ifluatex}
\newif\ifxetexorluatex % a new conditional starts as false
\ifnum 0\ifxetex 1\fi\ifluatex 1\fi>0
   \xetexorluatextrue
\fi

\ifxetexorluatex
  \usepackage{fontspec}
\else
  \usepackage[T1]{fontenc}
  \usepackage[utf8]{inputenc}
  \usepackage[lighttt]{lmodern}
  \ttfamily\DeclareFontShape{T1}{lmtt}{m}{it}{<->sub*lmtt/m/sl}{}
\fi

\usepackage[english,magyar]{babel} % Alapértelmezés szerint utoljára definiált nyelv lesz aktív, de később külön beállítjuk az aktív nyelvet.

\usepackage{emptypage} % omit page number on empty pages

%\usepackage{cmap}
\usepackage{amsfonts,amsmath,amssymb} % Mathematical symbols.
\DeclareMathOperator*{\argmin}{arg\,min}
\usepackage[ruled,boxed,resetcount,linesnumbered]{algorithm2e} % For pseudocodes. % beware: this is not compatible with LuaLaTeX, see http://tex.stackexchange.com/questions/34814/lualatex-and-algorithm2e
\usepackage{booktabs} % For publication quality tables for LaTeX
\usepackage{graphicx}

%\usepackage{fancyhdr}
%\usepackage{lastpage}

\usepackage{geometry}
%\usepackage{sectsty}
\usepackage{setspace} % For setting line spacing

\usepackage[unicode]{hyperref} % For hyperlinks in the generated document.
\usepackage{xcolor}
\usepackage{listings} % For source code snippets.

\usepackage[amsmath,thmmarks]{ntheorem} % Theorem-like environments.

\usepackage[hang]{caption}

\singlespacing

\newcommand{\selecthungarian}{
	\selectlanguage{magyar}
	\setlength{\parindent}{2em}
	\setlength{\parskip}{0em}
	\frenchspacing
}

\newcommand{\selectenglish}{
	\selectlanguage{english}
	\setlength{\parindent}{0em}
	\setlength{\parskip}{0.5em}
	\nonfrenchspacing
	\renewcommand{\figureautorefname}{Figure}
	\renewcommand{\tableautorefname}{Table}
	\renewcommand{\partautorefname}{Part}
	\renewcommand{\chapterautorefname}{Chapter}
	\renewcommand{\sectionautorefname}{Section}
	\renewcommand{\subsectionautorefname}{Section}
	\renewcommand{\subsubsectionautorefname}{Section}
}

\usepackage[numbers]{natbib}
\usepackage{xspace}

\usepackage[colorinlistoftodos,prependcaption,textsize=tiny]{todonotes}

\usepackage{tikz-qtree-compat}
\usepackage{tikz}
\usetikzlibrary{shapes.misc}
\usetikzlibrary{decorations.markings}
\usetikzlibrary{positioning}
\tikzset{cross/.style={cross out, draw, 
         minimum size=2*(#1-\pgflinewidth), 
         inner sep=0pt, outer sep=0pt}}

\usepackage{subcaption}