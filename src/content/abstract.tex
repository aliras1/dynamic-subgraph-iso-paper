\pagenumbering{roman}
\setcounter{page}{1}

\selecthungarian

%----------------------------------------------------------------------------
% Abstract in Hungarian
%----------------------------------------------------------------------------
\chapter*{Kivonat}\addcontentsline{toc}{chapter}{Kivonat}

A gráf- és a részgráf-izomorfia vizsgálatának számos alkalmazási területe van kezdve a szociális hálózatok analíizisétől az áramkörök tervezéséig. A gráfok csúcspontokból és csúcspontokat összekötő élekből állnak. A részgráf-izomorfia azt jelenti, hogy G1 és G2 gráfokat nézve van-e olyan G2’ részgráfja G2-nek, amelyik izomorf G1-gyel, azaz létezik egy-egyértelmű megfeleltetés (bijekció) G2’ és G1 csúcsai és élei között. Ez a probléma NP-teljes, vagyis nem ismerünk polinomiális idejű algoritmust a megoldására. A problémára azonban számos state-of-the-art megoldás létezik, melyek különböző módokon próbálják minimalizálni az adott eljárás lépésszámát.

A valóságban léteznek olyan gráfok is, melyek időben változhatnak. Folyamatosan új csúcsok, élek keletkezhetnek és törlődhetnek. A dolgozat célja a részgráf-izomorfia vizsgálata ilyen dinamikusan változó gráfok esetén. 


\vfill
\selectenglish


%----------------------------------------------------------------------------
% Abstract in English
%----------------------------------------------------------------------------
\chapter*{Abstract}\addcontentsline{toc}{chapter}{Abstract}

Graph- and subgraph-isomorphism is used for a wide range of problems such as social network analysis and circuit board design. A graph consists of vertices and edges that connect two distinct vertices. Subgraph-isomorphism means that given G1 and G2 graphs, a G2' subgraph of G2 exists such that it is isomorphic with G1. In other words, there exists a bijection between the vertices of G2' and G1. This is a well-known NP-hard problem. Several state-of-the-art algorithms exist that optimize their workings in different ways in order to accelerate their performance.

In reality, however, not all graphs are static in time. Some graphs are chaning constantly; new vertices and edges are created and deleted on a regular basis. The goal of this work is the analysis of subgraph-isomorphism in case of dynamic graphs. 


\vfill
\cleardoublepage

\selectthesislanguage

\newcounter{romanPage}
\setcounter{romanPage}{\value{page}}
\stepcounter{romanPage}