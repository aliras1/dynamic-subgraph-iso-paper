\chapter{Incremental algorithms}

This chapter describes two incremental algorithms for finding subgraph isomorphisms
in dynamic graphs. First we introduce the approach proposed in []. Then we present
our generic method which can be applied to DAF and VF2++ respectively.



\section{Locality based method}

Let \(d\) denote the diameter of the query graph \(q\) which corresponds to the 
length of the longest shortest path in $q$. Let $\Delta e$ denote the addition or
deletion of edge $e = (v, v')$ in $G$. Moreover let $V(d, e)$ be the set of vertices
in $G$ that are within a distance $d$ from $v$ or $v'$. Finally, let $G(d, e)$
be the subgraph of $G$ induced by $V(d, e)$, and $\Delta G(d, e)$ be the subgraph
of $\Delta G$ induced by $V(d, e)$, where $\Delta G = (V, E \setminus \{e\})$ or
$\Delta G = (V, E \cup \{e\})$ depending on the operation of $\Delta e$. With these
notations, we can define the locality property of subgraph isomorphism. For any 
given changes of $\Delta e$ in $G$, the changes $\Delta M$ in the set of mappings
$M(q, G)$ is the difference between $M(q, G(d, e))$ and $M(q, \Delta G(d, e))$, i.e.
the difference between the mappings found in the $d$th neighborhood of the original graph
and in the neighborhood's modified version. This is a trivial property because new mappings can 
appear or become obsolete only in the immediate vicinity of the affected edge $e$. 
Mappings further away from $e$ than $d$ are not affected by $\Delta e$ because neither 
vertices of $e$ are reachable from there. Regardless of that a new edge was added
or an old one was removed, mappings can both appear and become obsolete in the updated
graph. The mappings that have to be added to the existing set is 
$M(q, \Delta G(d, e)) \setminus M(q, G(d, e))$, while the mappings that have to be
removed are $M(q, G(d, e)) \setminus M(q, \Delta G(d, e))$.

The algorithm uses this locality property as follows. Given a query graph $q$ and a 
data graph $G$, compute the initial set of mappings $M$ with a classic subgraph 
isomorphism algorithm, e.g. VF2++. Then, for each update $\Delta e$, (1) find the 
diameter $d$ of $q$, (2) extract the subgraph $\Delta G(d, e)$ from $G$, (3) compute 
$M(q, \Delta G(e, d))$, (4) then update $M$ as described above.

\SetKwComment{Comment}{/* }{ */}

\begin{algorithm}
\caption{Locality algorithm}\label{alg:two}
\KwIn{$q, G, \Delta e, M = VF2++(q, G)$}
$d \gets diameter(q)$\;
$G_{d, e} \gets G(d, e)$\;
$\Delta M \gets VF2++(q, G_{d, e})$\;
$M = M \cup M(q, \Delta G(d, e)) \setminus M(q, G(d, e))$\;
$M = M \setminus M(q, G(d, e)) \setminus M(q, \Delta G(d, e))$\;
\KwRet{$M$}
\end{algorithm}

Instead of re-computing all mappings in $G$, this method works on a subset of $G$.
Although a regular subgraph isomorphism is running in the background, reducing the
size of the input data graph can make this variant faster. The effectiveness depends
on the query graph and the data graph. If $G(d, e)$ remains small compared to $G$,
there will be a massive speed up. However if $G(d, e) \sim G$, for example in case of
small world networks, the algorithm performs the same as a regular subgraph isomorphism
algorithm.


\section{Search tree based method}

This chapter describes our method of making the previously introduced algorithms
incremental. One way or another, both algorithms traverse a search tree eventually.
In both cases the search tree is purely abstract, it has no physical manifestation
in the memory whatsoever. We propose to store the traversed search tree while running
subgraph isomorphism initially, and make use of it in future updates. $T_{q, G}$ is a
search tree of query graph $q$ and data graph $G$, where paths from the root to a leaf
correspond to (partial) mappings, and a node contains an $(u, v)$ pair where 
$u \in V_q, v \in V_G$ denotes a single vertex mapping. The root $r$ of $T_{q, G}$
contains an empty mapping. A root-to-leaf path, whose length $l = |V_q|$ is a complete
mapping because it maps all nodes of $q$. The rest of the root-to-leaf paths correspond
to partial mappings.

In a typical scenario at the end of an execution of a subgraph isomorphism algorithm,
the search tree contains some complete mappings and orders of magnitude more partial 
mappings that for some reason could not be extended any further. These partial mappings 
play a crucial role in the incremental version. One can think about them as potential 
partially precalculated mappings. The reason, why a partial mapping could not be extended
is because the algorithm ran out of valid candidates in the area of the mapped nodes' 
neighborhood in \(G\). This is caused because by some kind of conflict between the 
topology of \(q\) and the topology of the mapped nodes and their neighborhood. These 
conflicts could be resolved when a new edge is added to or an old one is deleted from 
\(G\). At this point, having these partial mappings in memory can speed up the process 
of removing  matches that became obsolete and finding new ones that were impossible in 
the past.

In the upcoming sections, we define an incremental algorithm for each type of graph 
modifications.

\subsection{Node deletion}

Deleting a node from $v_d$ from $G$ can only reduce the original number of mappings.
Mappings that did not contain $v_d$ are unaffected, while those that contained $v_d$ 
have to be removed since they no longer can map to $v_d$. Now we only have to think 
through that no new mappings could arise. Indeed, that cannot be the case because
the only edges that were removed are edges belonging to $v_d$ which means only 
partial mappings which contain $v_d$ are affected. However since $v_d$ itself is also
deleted, these mappings become obsolete, which means that no new mappings can be
found.

Since there are no new mappings to be found, the only work to do in case of a node
deletion is pruning the search tree $T_{q, G}$. More specifically, cut off every
sub-branch that starts with a node $(u, v)$ where $v = v_d$. In worst case scenario
this would mean traversing the whole search space. We already saw however, that any
given change can only affect mappings whose vertices are in the $d$th neighborhood
$N_d(G, v_d)$ of $v_d$. Thus we can skip any branches that map a query graph vertex
to a node $v \notin N_d(G, v_d)$.

\begin{algorithm}[htp]
    \SetAlgoLined\DontPrintSemicolon
    \SetKwFunction{proc}{delete\_node}
    \KwIn{$q, G, T_{q, G}, v_d$}
    \nl $d \gets $ diameter($q$)\;
    \nl $N_d(G, v_d) \gets$ $d$th neighborhood of $v_d$ in $G$\;
    \SetKwProg{myproc}{Procedure}{}{}
    \myproc{\proc{node, $v_d$}}{
        \ForEach(){child $\in$ node.children}{
            \nl $(u, v) \gets $ child.mapping\;
            
            \uIf{$v = v_d$}{
                \nl node.remove(child)\;
            }
            \ElseIf(){$v \in N_d(G, v_d)$}{
                \nl \proc{child, $v_d$}\;
            }
        }
    }
    \caption{Delete node incrementally}
\end{algorithm}

\subsection{Node insertion}

Inserting a new node has no effects on mappings whatsoever because at the time of insertion,
it cannot be part of any mappings because it has not connections. Thus inserting a new node
needs no special care.

\subsection{Edge operations}

In case of induced subgraph isomorphism, both inserting and deleting an edge can result 
in loosing and gaining mappings. Removing an edge $e$ makes mappings where nodes of $e$
were part of the mapping no longer valid. At the same time, in case of partial mappings
where $e$ made it impossible to further extend, removing it may cause new mappings to
appear. For example in \ref{fig:insdelmap}, as a result of removing the BD edge from $G$,
$M(q_1, G')$ will loose half of its mappings, while $M(q_2, G')$ will double the amount of
its mappings. 

\begin{figure}[h]
    \centering
    \begin{subfigure}{.2\textwidth}
      \centering
      \begin {tikzpicture}[auto, node distance=1cm,thick,main node/.style={circle,draw}]
        \node[main node] (A) {\scriptsize$A$};
        \node[main node] (B) [below=of A] {\scriptsize$B$};
        \node[main node] (C) [right =of B] {\scriptsize$C$};
        \node[main node] (D) [right =of A] {\scriptsize$D$};    
        \draw (A) -- (B);
        \draw (B) -- (C);
        \draw (C) -- (D);
        \draw (D) -- (A);
        \draw (A) -- (C);
        \draw[red] (B) -- (D);    
      \end{tikzpicture}
      \caption{$G$}
      \label{fig:sfig1}
    \end{subfigure}%
    \begin{subfigure}{.2\textwidth}
      \centering
      \begin {tikzpicture}[auto, node distance=1cm,thick,main node/.style={circle,draw}]
        \node[main node] (A) {\scriptsize$1$};
        \node[main node] (B) [below=of A] {\scriptsize$2$};
        \node[main node] (C) [right =of B] {\scriptsize$3$};
        \node[main node] (D) [right =of A] {\scriptsize$4$};    
        \draw (A) -- (B);
        \draw (B) -- (C);
        \draw (C) -- (D);
        \draw (D) -- (A);
        \draw (B) -- (D);    
      \end{tikzpicture}
      \caption{$q_1$}
      \label{fig:sfig2}
    \end{subfigure}
    \begin{subfigure}{.2\textwidth}
        \centering
        \begin {tikzpicture}[auto, node distance=1cm,main node/.style={circle,draw}]
          \node[main node] (A) {\scriptsize $1$};
          \node[main node] (B) [below=of A] {\scriptsize $2$};
          \node[main node] (C) [right =of B] {\scriptsize $3$};
          \node[main node] (D) [right =of A] {\scriptsize $4$};    
          \draw (A) -- (B);
          \draw (B) -- (C);
          \draw (C) -- (D);
          \draw (D) -- (A);
        \end{tikzpicture}
        \caption{$q_2$}
        \label{fig:sfig3}
    \end{subfigure}
    \begin{subfigure}{.2\textwidth}
        \centering
        \begin {tikzpicture}[auto, node distance=1cm,main node/.style={circle,draw}]
          \node[main node] (A) {\scriptsize $1$};
          \node[main node] (B) [below=of A] {\scriptsize $2$};
          \node[main node] (C) [right =of B] {\scriptsize $3$};
          \draw (A) -- (B);
          \draw (B) -- (C);
          \draw (C) -- (A);
        \end{tikzpicture}
        \caption{$q_2$}
        \label{fig:sfig4}
    \end{subfigure}
    \caption{Example how mappings can appear and disappear in both cases of edge deletion and insertion.}
    \label{fig:insdelmap}
\end{figure}

The same goes for edge insertion. If we were to insert the BD edge into $G$ then
$M(q_1, G')$ would find new mappings, while $M(q_2, G')$ should remove the half 
of them. It is clear that these can happen simultaneously, as well.

\subsubsection{Edge deletion}

When an edge $e = (v_1, v_2)$ is removed from $G$, first, we have to prune the search
tree to remove mappings that are no longer valid and we have to somehow find the new 
mappings. We have to prune all branches that correspond to a partial mapping which both
$v_1$ and $v_2$ are part of, and there is an edge between $m^{-1}(v_1)$ and $m^{-1}(v_2)$ 
in $q$. We find all paths that correspond to such partial mappings, and prune them down
from the first point where $v_1$ and $v_2$ both appeared in the path. \ref{alg:deledgeprune}
shows the pruning algorithm when an edge was deleted.

\begin{algorithm}[htp]
    \SetAlgoLined\DontPrintSemicolon    
    \SetKwFunction{prune}{prune}
    
    \SetKwProg{pruneproc}{Procedure}{}{}
    \pruneproc{\prune{node, $v_1$, $v_2$}}{
        \nl $(u, v) = $ node.mapping\;
        \uIf(){$v_1$ not marked as found $\wedge$ $v = v_1$}{
            \nl mark $v_1$ as found\;
        }
        \ElseIf(){$v_2$ not marked as found $\wedge$ $v = v_2$}{
            \nl mark $v_2$ as found\;
        }

        \If{$v_1$ is marked $\wedge$ $v_2$ is marked}{
            \If{$(m^{-1}(v_1), m^{-1}(v_2)) \in E_q$}{
                \nl \KwRet{true}\;
            }
        }

        \nl $node.children = \{child \in node.children: \neg \prune(child, v_1, v_2) \}$\;     
    }    
    \caption{Prune search tree on edge deletion}
    \label{alg:deledgeprune}
\end{algorithm}

Fig \ref{fig:pruneex} shows an example partial search tree of $G$ and $q_3$ from Fig. \ref{fig:insdelmap}.
The red nodes represent the pruned sub-branches as a result of removing the $BD$ edge from $G$.

\begin{figure}[h]
    \centering    
    \begin {tikzpicture}[auto, node distance=1cm,main node/.style={circle,draw}]
        \node[main node] (S) {\scriptsize $\emptyset, \emptyset$};
        \node[main node] (R) [below=of S] {\scriptsize $1, B$};
        \node[main node] (A) [below left=of R,red] {\scriptsize $2, D$};
        \node[main node] (AA) [below left=of A,red] {\scriptsize $3, A$};    
        \node[main node] (AB) [below right=of A,red] {\scriptsize $3, C$};    
        \node[main node] (B) [below right=of R] {\scriptsize $2, A$};    
        \node[main node] (BB) [below right=of B] {\scriptsize $3, C$};    
        \draw (S) -- (R);
        \draw (R) -- (A);
        \draw (R) -- (B);
        \draw (A) -- (AA);
        \draw (A) -- (AB);        
        \draw (B) -- (BB);
    \end{tikzpicture}
    \caption{Pruning of a partial search tree of $G$ and $q_3$ from Fig. \ref{fig:insdelmap}.}
    \label{fig:pruneex}
\end{figure}

Finding new mappings is somewhat harder because we have to extend the search tree instead
of just pruning it. Because of the locality property of the incremental subgraph isomorphism
problem, we know for sure that any new mappings that will appear have to contain either $v_1$
or $v_2$ or both. Thus our job is reduced to look for partial mappings that contain either of 
those two vertices. There are two kinds of partial mappings to consider. Typically, the search
tree will contain some partial mappings that contain one of these two vertices but definitely
not all of them. As a first step, we add these partial mappings to the search tree. We traverse 
the tree and check if a node has a child that maps the next vertex $u \in q$ to $v_i$. If it does
not, and $v_i$ was not covered already in the mapping, and it causes no conflicts between the 
topology of the mapping and $q$ then we create a new child under the current node with 
$m(u) = v_i$. Now that we have all the partial mappings containing $v_1$ or $v_2$ - note that not 
all of them are expanded yet - we find all those leaves whose path from the root corresponds to a 
partial mapping that contains $v_1$ or $v_2$ and we execute our choice of subgraph isomorphism 
algorithm from that point. Naturally, these two steps can be combined into one traversal. If we 
hit a node where (1) we can create a new child or (2) a node is a leaf and its path contains $v_1$ 
or $v_2$, and it is not a complete mapping, we execute our subgraph isomorphism algorithm. We
can speed up the algorithm by only considering those branches that contain vertices such that
$\forall v \in m: v \in N_d(G, v_1) \cup N_d(G, v_2)$, i.e. branches that contain only mapped 
vertices such that they are in the $d$th neighborhood of $e$. Alg. \ref{alg:deledgefind}
formally describes the algorithm of finding new mappings after an edge has been deleted, and
Alg. \ref{alg:deledge} shows the incremental algorithm for subgraph isomorphism in case of edge 
deletion.

\begin{algorithm}[ht]
    \SetAlgoLined\DontPrintSemicolon    
    \SetKwFunction{find}{find\_mappings}
    
    \SetKwProg{findproc}{Procedure}{}{}
    \findproc{\find{node, forced\_candidates}}{
        \uIf{node.depth $=$ $|V_q|$ $\lor$ node.mapping.v $\notin N_d(G, (v_1, v_2))$}{
            \nl \KwRet{}\;
        }        

        \nl u $\gets$ node.extendable\_vertex\_of\_children\;
        \ForEach(){$v_c \in forced\_candidates$}{
            \uIf{$\nexists child: child.mapping = (u, v_c)$}{
                \uIf(){$v_c$ is not covered $\wedge$ $v_c$ is good candidate}{
                    \nl child = node.add\_child(u, $v_c$)\;
                    \nl ISO(child)\;
                }
            }
        }
        
        \uIf(){node is leaf $\land$ $(v_1 \in m \lor v_2 \in m)$}{
            \nl ISO(node)\;
        }
        \Else{
            \ForEach(){child $\in$ node.children}{
                \nl \find(child, forced\_candidates)\;
            }
        }

    }        
    \caption{Find new mappings incrementally}
    \label{alg:deledgefind}
\end{algorithm}

\begin{algorithm}[ht]
    \SetAlgoLined\DontPrintSemicolon    
    \SetKwFunction{prune}{prune}
    \SetKwFunction{find}{find\_mappings}
    \KwIn{$q, G, T_{q, G}, e$}
    
    \nl $d \gets $ diameter($q$)\;
    \nl $(v_1, v_2) = e$\;
    \nl $N_d(G, (v_1, v_2)) \gets$ $d$th neighborhood of $v_1$ and $v_2$ in $G$\;
    \nl \prune(root($T_{q, G}$), $v_1, v_2$)\;
    \nl \find(root($T_{q, G}$), $\{v_1, v_2\}$)\;    
    
    \caption{Delete edge incrementally}
    \label{alg:deledge}
\end{algorithm}


\subsubsection{Edge insertion}

Inserting a new edge into $G$ will have a really similar effect on the mappings as and edge
deletion. Mappings can both appear and disappear from the original mapping set. In fact, the
algorithm described above is almost applicable to this problem out of the box. It is clear 
that finding new mappings will be the same in this case, as well. We have to take care about
how we prune the search tree before that. Indeed, removing nodes that pass the test 
$(m^{-1}(v_1), m^{-1}(v_2)) \in E_q$ makes no sense because we just added that new edge. We
can make this condition generic however, so that both edge operations can use the same predicate.
Instead of separating the two cases, we will simply check that an edge between $v_1$ and $v_2$
in $G$ exists if and only if an edge between $m^{-1}(v_1)$ and $m^{-1}(v_2)$ in $q$ exists.
With these modifications, the final pruning algorithm can be found in Alg. \ref{alg:edgeprune}.

\begin{algorithm}[ht]
    \SetAlgoLined\DontPrintSemicolon    
    \SetKwFunction{prune}{prune}
    
    \SetKwProg{pruneproc}{Procedure}{}{}
    \pruneproc{\prune{node, $v_1$, $v_2$}}{
        \nl $(u, v) = $ node.mapping\;
        \uIf(){$v_1$ not marked as found $\wedge$ $v = v_1$}{
            \nl mark $v_1$ as found\;
        }
        \ElseIf(){$v_2$ not marked as found $\wedge$ $v = v_2$}{
            \nl mark $v_2$ as found\;
        }

        \If{$v_1$ is marked $\wedge$ $v_2$ is marked}{
            \nl is\_edge\_in\_q $\gets$ $(m^{-1}(v_1), m^{-1}(v_2)) \in E_q$\;
            \nl is\_edge\_in\_G $\gets$ $(v_1, v_2) \in E_G$\;
            \If{is\_edge\_in\_q $\neq$ is\_edge\_in\_G}{
                \nl \KwRet{true}\;
            }
        }

        \nl $node.children = \{child \in node.children: \neg \prune(child, v_1, v_2) \}$\;     
    }    
    \caption{Prune search tree on edge operation}
    \label{alg:edgeprune}
\end{algorithm}